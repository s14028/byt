\documentclass[a4page]{article}

\usepackage[polish]{babel}
\usepackage[utf8]{inputenc}
\usepackage{polski}
\usepackage[T1]{fontenc}
\frenchspacing
\usepackage{indentfirst}
\usepackage{pdfpages}
\usepackage{graphicx}
\usepackage{changepage}

\begin{document}
	\begin{titlepage}
		\title{Tabela do sklepu muzycznego ,,MusicPro --- online''\\}
		\author{Kamil Jurczuk \\ Bartosz Lodziński \\ Illia Shkroba \\ Piotr Wezgraj \\Paweł Wiszniewski}
		\makeatletter
			\centering
			{
				{\Huge{\@title}}
				\vspace{5cm}
				{\LARGE{\@author}}
			}
		\makeatother

	\end{titlepage}

	\begin{table}
	\newcounter{rownumber}
	\newcommand\next{\stepcounter{rownumber}\arabic{rownumber}}
		\begin{adjustwidth}{-4.5cm}{}
			\begin{tabular}{|c|c|c|c|}
			
				\hline
					Lp & Grupy czasownikowe & Usługa (przypadek użycia) & Operacja lub odpowiedzialność\\
				\hline
					\next & \textbf{tworzymy} sklep & a & a\\
				\hline
					\next & można \textbf{kupować} instrumenty & a & a\\
				\hline
					\next & stan \textbf{opisuje} liczbę sztuk & a & a\\
				\hline
					\next & stan \textbf{nie może być} ujemny & a & a\\
				\hline
					\next & dostępny stan \textbf{odwzorowuje} posiadania instrumentu & a & a\\
				\hline
					\next & \textbf{istnieje} tylko jeden szef & a & a\\
				\hline
					\next & \textbf{nie będzie} dwóch szefów & a & a\\
				\hline
					\next & pracownik \textbf{realizuje} zamówienie & realizacja zamówienia & a\\
				\hline
					\next & pracownik \textbf{nadzorowuje} zamówienie & a & a\\
				\hline
					\next & pracownik \textbf{sprawdza} czy opłacone & a & a\\
				\hline
					\next & pracownik \textbf{zmienia} status zamówienia & a & a\\
				\hline
					\next & premię się \textbf{wylicza} ze wzoru $f(x) = 0.1 * x$& a & a\\
				\hline
					\next & system \textbf{przechowuje} informację o produktach & a & a\\
				\hline
					\next & instrument \textbf{ma} nazwę & a & a\\
				\hline
					\next & firma \textbf{może produkować} więcej jednego instrumentu & a & a\\
				\hline
					\next & system \textbf{przechowuje} dane klientów & a & a\\
				\hline
					\next & klient \textbf{musi założyć} konto & założenie konta & a\\
				\hline
					\next & klient \textbf{wypełnia} formularz aby \textbf{złożyć} zamówienie & składanie zamówienia & a\\
				\hline
					\next & system \textbf{przechowuje} dane pracowników & a & a\\
				\hline
					\next & pracownik \textbf{realizuje} $[1; 2]$ funkcji & a & a\\
				\hline
					\next & \textbf{nie przewidujemy} rozszerzenia listy odpowiedzialności & a & a\\
				\hline
					\next & klient \textbf{składa} zamówienie & a & a\\
				\hline
					\next & klient \textbf{może wybrać} poprzez formularz & a & a\\
				\hline
					\next & klient \textbf{wybiera} formę dostawy & a & a\\
				\hline
					\next & klient \textbf{wybiera} formę opłaty & a & a\\
				\hline
					\next & zamówienie \textbf{obsługują} co najmniej dwóch pracowników & a & a\\
				\hline
					\next & \textbf{rejestrujemy} pracowników realizujących zamówienia & a & a\\
				\hline
					\next & klient \textbf{może wybrać} poprzez formularz & a & a\\
				\hline
	
			\end{tabular}
		\end{adjustwidth}
	\end{table}

	\newpage

	\begin{table}
		\large

		\begin{tabular}{|c|c|c|c|}
		
			\hline
				Lp & Grupy rzeczownikowe & Kandydat na obiekt & Kandydat na atrybut\\
			\hline
				1 & a & a & a\\
			\hline

		\end{tabular}
	\end{table}
\end{document}
